\documentclass{standalone}
  \usepackage[utf8]{inputenc}
\usepackage{tikz}
\usepackage{xifthen}


\RequirePackage[T1]{fontenc}
\RequirePackage{times}

\renewcommand*{\familydefault}{\sfdefault}
\RequirePackage{amsmath,amsthm, amssymb, latexsym}



\usetikzlibrary{fit,arrows.meta}
\begin{document}

\boldmath
\bfseries
\Large

\tikzstyle{arr} = [-{Latex[length=2mm,width=2mm]}]
\tikzstyle{neuron}=[circle,fill=black!25,minimum size=17pt,inner sep=0pt]
\tikzstyle{hidden neuron}=[neuron, fill=blue!50]
\tikzstyle{output neuron}=[neuron, fill=red!50]
\tikzstyle{annot} = [text width=10em, text centered]
\def\layersep{2.5cm}
\pagestyle{empty}

\begin{tikzpicture}[shorten >=1pt,->,arr, node distance=\layersep]


    % Draw the first hidden layer
    \foreach \name / \x in {1,...,2}
    % This is the same as writing \foreach \name / \y in {1/1,2/2,3/3,4/4}
        \node[hidden neuron] (H1-\name) at (\x*2,0) {};

    % Draw the second hidden layer nodes
    \foreach \y in {1,...,3}
        \path[xshift=-1cm]
            node[hidden neuron] (H2-\y) at (\y*2,-\layersep) {};

    % Draw the output layer node
    \foreach \x in {1,...,2}
	    \node[output neuron] at (2*\x,-2*\layersep) (O-\x) {};


    % Connect every node in the input layer with every node in the
    % hidden layer.
    \foreach \source in {1,...,2}
        \foreach \dest in {1,...,3}
        	\ifthenelse{\NOT 2=\dest \OR \NOT 2 = \source}{\path (H1-\source) edge (H2-\dest);}{};
            


    % Connect every node in the second hidden layer with the output layer
    \foreach \source in {1,...,3}
    	\foreach \dest in {1,...,2}
    	\ifthenelse{\NOT 1=\source \OR \NOT 2 = \dest}{\path (H2-\source) edge (O-\dest);}{};
        



    \node[annot, xshift=-1cm,below of=O-2,node distance=1cm] (vis_effect) {\Large Visible effects};
    \node[annot,above of=H1-1,xshift=1cm,node distance=1cm] {\Large Hidden causes};
\end{tikzpicture}
% End of code
\end{document}

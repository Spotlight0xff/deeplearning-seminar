\section{Sigmoid Belief Networks}
\label{sec:sbn}
\begin{figure}[htb]
\centering
%& /home/spotlight/.config/TikzEdtWForms/TikzEdtWForms/0.2.1.0/temp_header
\documentclass{standalone}
\usepackage{pgfplots}
\pgfplotsset{compat=1.12}

\begin{document}
\usetikzlibrary{arrows.meta,calc,fit}
\usetikzlibrary{backgrounds}
\tikzstyle{h_unit} = [circle, draw, fill=blue!20, node distance=3cm, text width=2.5em, text centered]
\tikzstyle{h_text} = [rectangle, fill=blue!20, text centered]
\tikzstyle{o_unit} = [circle, draw, fill=red!20, node distance=3cm, text width=2.5em, text centered]
\tikzstyle{o_text} = [rectangle, fill=red!20, text centered]

\tikzstyle{arr} = [thick,-{Latex[length=3mm,width=3mm]}]
\begin{tikzpicture}


% first hidden layer
\node [h_unit] (hidden_l1_1) {};
\node [h_unit] (hidden_l1_2) at ([shift={(2,0)}] hidden_l1_1) {};
\node (hidden_l1_others) at ([shift={(1,0)}] hidden_l1_1) {\Large $\cdots$};

% second hidden layer
\node [h_unit] (hidden_l2_1) at ([shift={(-1.5,-2)}] hidden_l1_1){};
\node [h_unit] (hidden_l2_2) at ([shift={(1,-2)}] hidden_l1_1) {};
\node [h_unit] (hidden_l2_3) at ([shift={(3.5,-2)}] hidden_l1_1) {};
\node (hidden_l2_others_1) at ([shift={(1.25,0)}] hidden_l2_1) {\Large $\cdots$};
\node (hidden_l2_others_2) at ([shift={(1.25,0)}] hidden_l2_2) {\Large $\cdots$};

% "stochastic hidden causes"
\node [h_text] (text_hidden) at ([shift={(1,1)}] hidden_l1_1) {\Large hidden causes};

% observable layer
\node [o_unit] (observed_1) at ([shift={(-1,-4)}] hidden_l1_1){};
\node [o_unit] (observed_2) at ([shift={(4,0)}] observed_1) {};
\node (observed_others) at ([shift={(2,0)}] observed_1) {\Large $\cdots$};

% "observed variables"
\node [o_text] (text_observed) at ([shift={(1,-5)}] hidden_l1_1) {\Large observed variables};

% layer backgrounds:
% We are not showing the layer boundaries for now,
% because in the initial sigmoid belief nets there was no
% notion of layers (just ancestors)
\begin{pgfonlayer}{background}
  \node[] () {};
% \node[
%       fill=black!20,
%       node distance=4cm,
%       inner sep = 1em,
%       fit=(hidden_l1_1) (hidden_l1_2)
%       ] (layer_1){};
% 
% \node[
%       fill=black!20,
%       node distance=4cm,
%       inner sep = 1em,
%       fit=(hidden_l2_1) (hidden_l2_2) (hidden_l2_3)
%       ] (layer_2){};
\end{pgfonlayer}

% arrows from first hidden unit in first "layer"
\draw[arr] (hidden_l1_1) -- (hidden_l2_1);
\draw[arr] (hidden_l1_1) -- (hidden_l2_2);
\draw[arr,dashed] (hidden_l1_1) -- (observed_1);

% arrows from second hidden unit in first "layer"
\draw[arr] (hidden_l1_2) -- (hidden_l2_2);
\draw[arr] (hidden_l1_2) -- (hidden_l2_3);

% arrows from first hidden unit in second "layer"
\draw[arr] (hidden_l2_1) -- (observed_1);

% arrows from second hidden unit in second "layer"
\draw[arr] (hidden_l2_2) -- (observed_1);
\draw[arr] (hidden_l2_2) -- (observed_2);

% arrows from third hidden unit in second "layer"
\draw[arr] (hidden_l2_3) -- (observed_2);
\end{tikzpicture}
\end{document}

  \caption[Architecture of Sigmoid Belief Nets]{Conceptual architecture of belief nets}
  \label{fig:sbn_arch}
  \medskip
  \small
  Bayesian networks, also called belief nets are an acyclic directed graph where blue nodes indicate hidden variables and red nodes represent observed variables.
  Arrows indicate stochastic dependency.
  Even though most belief nets are divided into many layers in the context of deep learning, the original definition does not specify that and thus indicates the dashed arrow the possibility of inter-layer dependencies.
\end{figure}

Sigmoid Belief Networks (SBN) were described in 1992 by R.M. Neal \cite{neal:1992} representing a specific type of bayesian networks \cite{pearl:1985} shown in Figure \ref{fig:sbn_arch}
In the case of SBNs however, all states are binary and the activation function is the sigmoid function. They are one of the earliest neural networks used for generative modeling, predating all other presented models in this article.
%As described in the overview above, SBNs contain a number of binary states $s$, most commonly divided into many layers.
Each state $s_i$ is a random variable in the bayesian framework which means it can be observed, hidden or unknown representing different states of knowledge. As such it can also be interpreted with a probability distribution.
\begin{equation}
  \label{eq:sbn_node}
p(s_i) = \sigma\bigg(\sum_{j<i}W_{j,i}s_j+b_i\bigg)
\end{equation}
where 
\begin{conditions}
  \sigma(x) & $\frac{1}{1 + e^{-x}}$ (sigmoid function)\\
  W_{j,i} & connection weights from state $s_j$ to $s_i$\\
\end{conditions}

To draw samples from this network, it is possible to simply evaluate Equation~\ref{eq:sbn_node} for each visible node.
However prior to that, the SBN first needs to learn how to generate good samples which includes the \emph{learning} and the \emph{inference} problem.\\
Learning means in this context to adjust the weights and biases of the network to be more likely to generate data similar to samples from the given training dataset.
Meanwhile statistical inference is tasked to infer the states of the hidden nodes based on the states in the visible layer.\\

Due to each state being statistically depend on all ancestors, inference means to evaluate all possible configurations of every previous node.
Especially in deep networks, this kind of computation is infeasible as it requires computing exponentially many configurations.\\

% TODO: connection to DBN, layer-wise greedy training
% Hinton et al. (2006)

% DONE

%However in order to draw good samples from this network, the weights need to be adjusted to make it 
%However most other operations involving this network are intractable for most cases, like learning and performing inference.

%In probabilistic networks, we are mostly interested in the aspects: sampling, learning and inferring.
%Even though sampling can be done efficiently, the other operations are intractable in almost all cases.
%\paragraph{Sampling}
%can be done using ancestral sampling through the hidden layers evaluating \ref{eq:sbn_node}

%These random variables are interpreted in a bayesian way where each one may be observed, hidden (or latent) or unknown representing different states of knowledge or belief \cite{definetti:1974}.

%\newpage
%\subsection{Learning}
%\begin{figure}[htb]
%\centering
%\includestandalone[mode=buildnew]{media/sbn_node}
  %\caption{SBN node}
  %\label{fig:sbn_node}
%\end{figure}


%In its most common form, SBNs are comprised by a number of layers where each node in a layer is a sigmoid function of its ancestors:

%While sampling from these networks is very efficient, computing the inference $p(z|x)$ over the hidden units is intractable.
%% show why?

%There have been several attempts on solving the inference problem, using inference networks which have shown promising results.

\newpage

\section{Sigmoid Belief Networks}
\label{sec:sbn}
Sigmoid Belief Networks (SBN) have been proposed in 1992 by R.M. Neal \cite{neal:1992} representing a a specific type of bayesian networks \cite{pearl:1985}.
They are one of the earliest neural networks used for generative modelling, predating all other presented models in this article.

These random variables are interpreted in a bayesian way where each one may be observed, hidden (or latent) or unknown representing different states of knowledge or belief \cite{definetti:1974}.
\begin{figure}[htb]
\centering
%& /home/spotlight/.config/TikzEdtWForms/TikzEdtWForms/0.2.1.0/temp_header
\documentclass{standalone}
\usepackage{pgfplots}
\pgfplotsset{compat=1.12}

\begin{document}
\usetikzlibrary{arrows.meta,calc,fit}
\usetikzlibrary{backgrounds}
\tikzstyle{h_unit} = [circle, draw, fill=blue!20, node distance=3cm, text width=2.5em, text centered]
\tikzstyle{h_text} = [rectangle, fill=blue!20, text centered]
\tikzstyle{o_unit} = [circle, draw, fill=red!20, node distance=3cm, text width=2.5em, text centered]
\tikzstyle{o_text} = [rectangle, fill=red!20, text centered]

\tikzstyle{arr} = [thick,-{Latex[length=3mm,width=3mm]}]
\begin{tikzpicture}


% first hidden layer
\node [h_unit] (hidden_l1_1) {};
\node [h_unit] (hidden_l1_2) at ([shift={(2,0)}] hidden_l1_1) {};
\node (hidden_l1_others) at ([shift={(1,0)}] hidden_l1_1) {\Large $\cdots$};

% second hidden layer
\node [h_unit] (hidden_l2_1) at ([shift={(-1.5,-2)}] hidden_l1_1){};
\node [h_unit] (hidden_l2_2) at ([shift={(1,-2)}] hidden_l1_1) {};
\node [h_unit] (hidden_l2_3) at ([shift={(3.5,-2)}] hidden_l1_1) {};
\node (hidden_l2_others_1) at ([shift={(1.25,0)}] hidden_l2_1) {\Large $\cdots$};
\node (hidden_l2_others_2) at ([shift={(1.25,0)}] hidden_l2_2) {\Large $\cdots$};

% "stochastic hidden causes"
\node [h_text] (text_hidden) at ([shift={(1,1)}] hidden_l1_1) {\Large hidden causes};

% observable layer
\node [o_unit] (observed_1) at ([shift={(-1,-4)}] hidden_l1_1){};
\node [o_unit] (observed_2) at ([shift={(4,0)}] observed_1) {};
\node (observed_others) at ([shift={(2,0)}] observed_1) {\Large $\cdots$};

% "observed variables"
\node [o_text] (text_observed) at ([shift={(1,-5)}] hidden_l1_1) {\Large observed variables};

% layer backgrounds:
% We are not showing the layer boundaries for now,
% because in the initial sigmoid belief nets there was no
% notion of layers (just ancestors)
\begin{pgfonlayer}{background}
  \node[] () {};
% \node[
%       fill=black!20,
%       node distance=4cm,
%       inner sep = 1em,
%       fit=(hidden_l1_1) (hidden_l1_2)
%       ] (layer_1){};
% 
% \node[
%       fill=black!20,
%       node distance=4cm,
%       inner sep = 1em,
%       fit=(hidden_l2_1) (hidden_l2_2) (hidden_l2_3)
%       ] (layer_2){};
\end{pgfonlayer}

% arrows from first hidden unit in first "layer"
\draw[arr] (hidden_l1_1) -- (hidden_l2_1);
\draw[arr] (hidden_l1_1) -- (hidden_l2_2);
\draw[arr,dashed] (hidden_l1_1) -- (observed_1);

% arrows from second hidden unit in first "layer"
\draw[arr] (hidden_l1_2) -- (hidden_l2_2);
\draw[arr] (hidden_l1_2) -- (hidden_l2_3);

% arrows from first hidden unit in second "layer"
\draw[arr] (hidden_l2_1) -- (observed_1);

% arrows from second hidden unit in second "layer"
\draw[arr] (hidden_l2_2) -- (observed_1);
\draw[arr] (hidden_l2_2) -- (observed_2);

% arrows from third hidden unit in second "layer"
\draw[arr] (hidden_l2_3) -- (observed_2);
\end{tikzpicture}
\end{document}

  \caption{SBN architecture}
  \label{fig:sbn_arch}
  \medskip
  \small
  Sigmoid Belief Network described as a acyclic directed graph where blue nodes indicate hidden variables while red nodes represent observed variables.
  Arrows show stochastic dependency. The dashed arrow indicates that it is possible in the original definition (and in the formula below) to have inter-layer dependencies but they are usually not modelled to simplify the architecture.
\end{figure}

\subsection{Learning}
\begin{figure}[htb]
\centering
\includestandalone[mode=buildnew]{media/sbn_node}
  \caption{Sigmoid Belief Net  architecture}
  \label{fig:sbn_node}
\end{figure}
In its most common form, SBNs are comprised by a number of layers where each node in a layer is a sigmoid function of its ancestors:
$$
p(s_i) = \sigma\bigg(\sum_{j<i}W_{i,j}s_j+b_i\bigg)
$$

While sampling from these networks is very efficient, computing the inference $p(z|x)$ over the hidden units is intractable.
% show why?

There have been several attempts on solving the inference problem, using inference networks which have shown promising results.



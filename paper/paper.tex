\documentclass[twoside,11pt,a4paper]{article}

% packages %%%%%%%%%%%%%%%%%%%%%%%%%%%%%%%%%%%%%%%%%%%%%%%%%%%%%%%%%%%%%%%%%%%%
\usepackage{graphicx,curves,float,rotating}


\usepackage{amsmath, amssymb, latexsym}  % math stuff
\usepackage{amsfonts}
\usepackage{amsopn}                             % um mathe operatoren zu deklarieren
\usepackage[american]{babel}                     % otherwise use british or american
\usepackage{theorem}                            % instead of \usepackage{amsthm}
\usepackage{dcolumn}
\usepackage{hyperref}
\usepackage[utf8x]{inputenc}
\usepackage{tikz}
\usepackage{algpseudocode}
\usepackage{algorithm}

\newcommand*\Let[2]{\State #1 $\gets$ #2}



% @ environment %%%%%%%%%%%%%%%%%%%%%%%%%%%%%%%%%%%%%%%%%%%%%%%%%%%%%%%%%%%%%%%%
\usepackage{xspace}                             % context sensitive space after macros
\makeatletter
\DeclareRobustCommand\onedot{\futurelet\@let@token\@onedot}
\def\@onedot{\ifx\@let@token.\else.\null\fi\xspace}
\def\eg{{e.g}\onedot} \def\Eg{{E.g}\onedot}
\def\ie{{i.e}\onedot} \def\Ie{{I.e}\onedot}
\def\cf{{c.f}\onedot} \def\Cf{{C.f}\onedot}
\def\etc{{etc}\onedot} \def\vs{{vs}\onedot}
\def\wrt{w.r.t\onedot} \def\dof{d.o.f\onedot}
\def\etal{{et al}\onedot}
\def\zB{z.B\onedot} \def\ZB{Z.B\onedot}
\def\dh{d.h\onedot} \def\Dh{D.h\onedot}
% %%%%%%%%%%%%%%%%%%%%%%%%%%%%%%%%%%%%%%%%%%%%%%%%%%%%%%%%%%%%%%%%%%%%%%%%%%%%%%%



%%%%%%%%%%%%%%%%%%%%%%%%%%%%%%%%%%%%%%%%%%%%%%%%%%%%%%%%%%%%%%%%%%%%%%%%%%%%%%%
%
%
%	Macros fuer neue Umgebungen
%
%
%%%%%%%%%%%%%%%%%%%%%%%%%%%%%%%%%%%%%%%%%%%%%%%%%%%%%%%%%%%%%%%%%%%%%%%%%%%%%%%
\newcommand*{\Frac}[2]{\frac{\displaystyle #1}{\displaystyle #2}}
\newlength{\textwd}
\newlength{\oddsidemargintmp}
\newlength{\evensidemargintmp}
\newcommand*{\hspaceof}[2]{\settowidth{\textwd}{#1}\mbox{\hspace{#2\textwd}}}
\newlength{\textht}
\newcommand*{\vspaceof}[3]{\settoheight{\textht}{#1}\mbox{\raisebox{#2\textht}{#3}}}
\newcommand*{\PreserveBackslash}[1]{\let\temp=\\#1\let\\=\temp}

\newenvironment{deflist}[1][\quad]%
{  \begin{list}{}{%
      \renewcommand{\makelabel}[1]{\textbf{##1}\hfil}%
      \settowidth{\labelwidth}{\textbf{#1}}%
      \setlength{\leftmargin}{\labelwidth}
      \addtolength{\leftmargin}{\labelsep}}}
{  \end{list}}


\newenvironment{Quote}% Definition of Quote
{  \begin{list}{}{%
      \setlength{\rightmargin}{0pt}}
      \item[]\ignorespaces}
{\unskip\end{list}}


\theoremstyle{break}
\theorembodyfont{\itshape}
\theoremheaderfont{\scshape}

\newtheorem{Cor}{Corollary}
\newtheorem{Def}{Definition}
%\newtheorem{Def}[Cor]{Definition}



\newcolumntype{.}{D{.}{.}{-1}}


\pagestyle{headings}
\textwidth 15cm
\textheight 23cm
\oddsidemargin 1cm
\evensidemargin 0cm
%\parindent 0mm



\begin{document}


\pagestyle{empty}

\begin{center}

    RWTH Aachen University\\
    Chair of Computer Science 6 \\
    Prof. Dr.-Ing. Hermann Ney\\[6ex]
    Selected Topics in Human Language Technology and Pattern Recognition WS 16/17\\[12ex]

    \LARGE
    \textbf{Deep Directed Generative Models} \\[6ex]
    \textit{André Merboldt} \\[6ex]
    \textit{346703} \\[6ex]
    \today

    \vfill
    \Large Supervisor: Tobias Menne
\end{center}

% blank page
\newpage
\
\newpage


% contents
\pagestyle{headings}
\tableofcontents
\listoftables
\listoffigures
%\newpage
%\pagestyle{empty}
%\
\newpage
\pagestyle{headings}


%----------------------------------
% ABSTRACT
%----------------------------------
\section{Abstract}
\label{sec:abstract}
%One of the key aspects of human intelligence is the ability to construct accurate and generalized internal representations by observing high-dimensional data.
%We present several deep directed generative models aiming to learn rich and hierarchical latent structures, either implicitly or explicitly, as well as transforming them back into input space.

Deep directed generative models leverage the possibilities of deep learning to learn rich, hierarchical structures from datasets in a possibly unsupervised context.
These models can be used to understand high-dimensional, real-world data and transform them into compact representations. Similar to how humans learn to generalize from sensory input, we discuss several models which can learn compact, internal representational structures.
In this article, we will focus on two recently proposed architectures, variational autoencoders, and generative adversarial networks.


% maybe include:
% * unsupervised models are in high demand due to the massive increase in the amount of unlabelled data due to web, videos, internat in general, laboratory measurements, etc.


%Deep Learning has advanced at an incredible speed the last few years and has achieved human-level performance in some discrimination and classification tasks (object recognition, speech recognition).
%These supervised methods have the huge drawback of requiring a vast amount of labelled data which is often expensive or unfeasible to obtain.
%Learning hidden structure from unlabelled data known as unsupervised learning has had less of an impact recently.
%However, it can be argued that this method is most similar to human learning, in particular inferring information from observations and building an appropriate model from it.
%We will discuss and explore some models for performing this inference as well as generating new data from these representations.
%Building reasonable representation of the world from unlabelled data is arguably one of the key aspects of machine intelligence.
%However one of the key aspects to machine intelligence is learning a rich (?) model about the world and draw conclusions from it.
%The above mentioned tasks require huge amount of labelled data to perform well, which is one of the restrictions of these supervised learning mechanisms.
%Unsupervised learning has the potential to make use of the vast unlabelled data surrounding the world.
%This data can be used to learn representations


%In this seminar paper we provide an overview and introduction to directed generative models,
%which are able to produce data following a distribution (...?).
% more info what generative models try to achieve
%After we have introduced several architectures, we will explore sigmoid belief networks (SBN)
%as an early generative model.\\

%Then we will turn to more recent proposals, in particular variational autoencoders (VAE) and
%generative adversarial networks (GAN).
%Variants of GANs have been shown to produce images which are able to fool a human discriminator 40\% of the time (citation needed, LAPGAN? or DCGAN?).
%Additionally to exploring these models and theoretical foundation, we provide an intuitive and simple application of a GAN.







%----------------------------------
% Motivation
%----------------------------------
\section{Introduction}
\label{sec:introduction}

"What I cannot create, I do not understand." -- Richard Feynman \\\\\\



Directed generative models are a wide class of models used for generating samples from an unknown distribution.
Lets denote $p_{data}$ as the unknown real-world data distribution from which samples in the given dataset are drawn.
In order to do so, we define an approximated distribution $p_{model}$ from which we are able to sample.
Under the assumption that $p_{model}$ approximates $p_{data}$ accurately, samples from the true data distribution should have high likelihood under the approximation as well.\\\\

Generative modelling has several important application areas, which will be highlighted on below.\\
Deep learning techniques have provided great success in applying gradient-based optimization methods to deep neural networks to achieve impressive results in discriminative models.
These models require however large amounts of labelled data, which is often expensive or infeasible to obtain.
Generative models can be used to learn \emph{representational} structures from rich, unlabelled data by having less parameters than the input data and are thus forced to capture only essential information.\\
\newpage


%Prior to discussion individual models, we need to declare a few assumptions and definitions.
%Additionally we assume that samples $x$ are also conditionally dependent on some hidden structure $z$: $x \sim p(x|z)$. 
%\begin{equation}
  %z \sim p(z)\\
  %x \sim p(x|z)
%\end{equation}

%\paragraph{Sampling}
%means for any given $z$ to draw a sample $x$ from our model.
%Most commonly, sampling is the easiest requirement to fulfil.

%\paragraph{Learning}
%of the model parameters through maximum likelihood estimation generally involves $p(x)$ or its gradient.\\
%To get $p(x)$, we first need to marginalize our the latent variables.
%\begin{equation}
  %p(x) = \sum_h p(x|z)p(z)
%\end{equation}
%Unfortunately, in the case of deep learning with many layers this is intractable, as we need to evaluate all model configurations.
%via maximum likelihood generally involves the gradients of $p(x)$
%\paragraph{Inference}
% motivation / what we want exactly and why
%Building intelligent systems requires having accurate and generalized information about the environment in order to perform actions or reason about the world.
%Inferring structured and meaning information from unlabelled data, a process called unsupervised learning, remains a challenging task.

%However, due to the massive increases in availability of data emerging from the web as well as the ongoing progress in computational power there has been significant progress using this data.

%Generative modelling aims to generate new data similar to previously seen input, in a way which is sound with the internal structure of the data.

%% on building reasonable representations
The assumption that data can be explained using an underlying, simpler structure than the raw input data is important to handle the high-dimensional nature of most data source.
In order to understand how these statistical models help us to discover the data structure, we will take a look at a few examples of high-dimensional real world data.
\begin{itemize}
  \item \textbf{Images} consisting of millions of pixels can be interpreted as data where each pixel represents an own dimension.
  However, these dimensions are high correlated and structured, for example nearby pixels have with high probability similar values.
  Looking at images of handwritten digits, the data can be reduced to a handful variables which may include the digit number, stroke width, position or size.
  Statistical models help us discover these relations and reduce them to representations in lower dimensional space.
  %\item commonly have thousands of dimensions, but the underlying structure is often far simpler. For example are 28x28 images of handwritten digits 784-dimensional but assuming one-hot encoding the most important information - the digit - it can be explained using 10 dimensions. There are a few more dimensions, stroke width, cursive, position, size. But still far lower than 784. We will take a look at this example later on in \ref{sec:vae}.
  \item \textbf{Audio data} is omni-present in the real world, however most data sources can be transformed from raw audio samples to spoken words or musical notes without losing too much information for reconstruction.
  \item \textbf{...}
\end{itemize}
% cope with
%To develop systems learning from real world data we use statistical models which are able to infer the underlying structure from
%Real world data is often high-dimensional and noise and transforming this data into an compact representation is therefore an important and also challenging task to perform.
%Generative models are ideal for this task since it has to understand the underlying structure to generate similar data.
%Building this structure is called \textbf{representation learning} and is key to many generative models and how they perform.\\

%With this assumption in mind we can take a look at some examples of input data.




The assumption that $p(x)$ is dependent on some underlying structure can now be expressed as having the conditional probability $p(x|z)$ where the latent variables $z$ are the causal factors. This stochastic dependency can be represented in a graphical model shown in figure \ref{fig:dgm}.\\

\begin{figure}[htb]
\centering
\includegraphics{media/directed_graphical_model}
  \caption{Directed graphical model}
  \label{fig:dgm}
  \medskip
  \small
  Arrows show stochastic dependency.
\end{figure}

%Inverting $p(x|z)$ yields $p(z|x)$ which is one of the main tasks of some generative models, for example the VAE \ref{sec:vae}.


%After a short overview of different approaches and methods and an introductory example with the sigmoid belief net (SBN), we will take a closer look at two recent promising frameworks, the variational autoencoder (VAE \ref{sec:vae}) and the generative adversarial network (GAN \ref{sec:gan}).



%Generative models are one of the most promising ways to perform unsupervised learning (citation needed).
%More specifically, generative modelling assumes a hidden structure $z$ explaining the input data $x$.
%This assumption is reasonable given the following examples:\\




%Generative models have been proposed as one of the most promising approaches towards
%learning representations of real world data by some of the leading researchers (YannLeCun, OpenAI blogpost).

%In recent years supervised learning has yielded impressive results,
%however for these models to succeed huge amount of labelled data is needed.
%Unsupervised learning, that is learning from unlabelled data, has been expressed
%as one hurdles toward general articial intelligence (citation needed).
%Early approaches toward this problem however have shown that these problems are very hard to learn (intractable?).

%In generative models the approach is different in a way that
%we search for a good internal representation of the data.
%This resembles the way humans learn about the world where we incrementally build
%a model of the world.




%----------------------------------
% Overview of Directed Generative Models
%----------------------------------
\section{Overview of Directed Generative Models}
\label{sec:overview}


\paragraph{Variational Autoencoders (VAE) \cite{vae:2013}\cite{rezende:2014}}
\label{par:overview_vae}
Variational Autoencoders (VAE) use two networks which are trained jointly using approximated learned inference.
The objective function is aimed at building a reasonable simple latent space and approximating the intractable posterior distribution using variational inference.
It consists of the recognition and inference network which can be understood in the autoencoder jargon as probabilistic encoder and decoder.
Using variational bayesian methods the intractable posterior distribution can be approximated and learned using backpropagation and gradient-based methods.

%But in addition to regular autoencoders encourages the VAE the latent space to match a specific probability distribution.
%This encouragement is done using a variational lower bound on the distribution output by the encoder network.

%where the encoder takes input data and translates it into a latent structure while
%the decoder takes the latent structure and produces data which is as close as possible to the initial input data.

%In the case of VAEs, the distribution is intractable and we can only approximate the posterior distribution.
% intractable -> use neural network, which we can train
% --> no closed form possible
%We use the KL-Divergence to train the network.

% TODO read chapter 19, approximate inference
%In VAEs however, this




\paragraph{Generative Adversarial Networks\cite{gan:2014}}
\label{par:overview_gan}
GANs are a game theoretic approach to generative modelling
in which two networks compete against each other.
The discriminator $D$ is trying to distinguish the generated samples
by the generator $G$ from real world data.
The goal for $G$ however is to produce data as realistic as possible.\\
During the early phase of training, $G$ produces random noise which
is easy to differentiate from groundtruth data.
Both networks are trained in parallel, where both networks try to get better
at their respective objective.
In the best-case scenario is $G$ able to produce data indistinguishly from real world data in which
case the discriminator will only be able to guess correctly with a 50\% chance.
This plateau is called Nash equilibrium (citation needed).








%----------------------------------
% Sigmoid Belief Networks
%----------------------------------
\section{Sigmoid Belief Networks}
\label{sec:sbn}
\begin{figure}[htb]
\centering
%& /home/spotlight/.config/TikzEdtWForms/TikzEdtWForms/0.2.1.0/temp_header
\documentclass{standalone}
\usepackage{pgfplots}
\pgfplotsset{compat=1.12}

\begin{document}
\usetikzlibrary{arrows.meta,calc,fit}
\usetikzlibrary{backgrounds}
\tikzstyle{h_unit} = [circle, draw, fill=blue!20, node distance=3cm, text width=2.5em, text centered]
\tikzstyle{h_text} = [rectangle, fill=blue!20, text centered]
\tikzstyle{o_unit} = [circle, draw, fill=red!20, node distance=3cm, text width=2.5em, text centered]
\tikzstyle{o_text} = [rectangle, fill=red!20, text centered]

\tikzstyle{arr} = [thick,-{Latex[length=3mm,width=3mm]}]
\begin{tikzpicture}


% first hidden layer
\node [h_unit] (hidden_l1_1) {};
\node [h_unit] (hidden_l1_2) at ([shift={(2,0)}] hidden_l1_1) {};
\node (hidden_l1_others) at ([shift={(1,0)}] hidden_l1_1) {\Large $\cdots$};

% second hidden layer
\node [h_unit] (hidden_l2_1) at ([shift={(-1.5,-2)}] hidden_l1_1){};
\node [h_unit] (hidden_l2_2) at ([shift={(1,-2)}] hidden_l1_1) {};
\node [h_unit] (hidden_l2_3) at ([shift={(3.5,-2)}] hidden_l1_1) {};
\node (hidden_l2_others_1) at ([shift={(1.25,0)}] hidden_l2_1) {\Large $\cdots$};
\node (hidden_l2_others_2) at ([shift={(1.25,0)}] hidden_l2_2) {\Large $\cdots$};

% "stochastic hidden causes"
\node [h_text] (text_hidden) at ([shift={(1,1)}] hidden_l1_1) {\Large hidden causes};

% observable layer
\node [o_unit] (observed_1) at ([shift={(-1,-4)}] hidden_l1_1){};
\node [o_unit] (observed_2) at ([shift={(4,0)}] observed_1) {};
\node (observed_others) at ([shift={(2,0)}] observed_1) {\Large $\cdots$};

% "observed variables"
\node [o_text] (text_observed) at ([shift={(1,-5)}] hidden_l1_1) {\Large observed variables};

% layer backgrounds:
% We are not showing the layer boundaries for now,
% because in the initial sigmoid belief nets there was no
% notion of layers (just ancestors)
\begin{pgfonlayer}{background}
  \node[] () {};
% \node[
%       fill=black!20,
%       node distance=4cm,
%       inner sep = 1em,
%       fit=(hidden_l1_1) (hidden_l1_2)
%       ] (layer_1){};
% 
% \node[
%       fill=black!20,
%       node distance=4cm,
%       inner sep = 1em,
%       fit=(hidden_l2_1) (hidden_l2_2) (hidden_l2_3)
%       ] (layer_2){};
\end{pgfonlayer}

% arrows from first hidden unit in first "layer"
\draw[arr] (hidden_l1_1) -- (hidden_l2_1);
\draw[arr] (hidden_l1_1) -- (hidden_l2_2);
\draw[arr,dashed] (hidden_l1_1) -- (observed_1);

% arrows from second hidden unit in first "layer"
\draw[arr] (hidden_l1_2) -- (hidden_l2_2);
\draw[arr] (hidden_l1_2) -- (hidden_l2_3);

% arrows from first hidden unit in second "layer"
\draw[arr] (hidden_l2_1) -- (observed_1);

% arrows from second hidden unit in second "layer"
\draw[arr] (hidden_l2_2) -- (observed_1);
\draw[arr] (hidden_l2_2) -- (observed_2);

% arrows from third hidden unit in second "layer"
\draw[arr] (hidden_l2_3) -- (observed_2);
\end{tikzpicture}
\end{document}

  \caption{Conceptual architecture of belief nets}
  \label{fig:sbn_arch}
  \medskip
  \small
  Bayesian networks, also called belief nets can be described as an acyclic directed graph where blue nodes indicate hidden variables while red nodes represent observed variables.
  Arrows show stochastic dependency. The dashed arrow indicates that it is possible in the original definition, even though most commonly the hidden variables are divided into layers.
\end{figure}

Sigmoid Belief Networks (SBN) have been proposed in 1992 by R.M. Neal \cite{neal:1992} representing a specific type of bayesian networks \cite{pearl:1985}.
Figure \ref{fig:sbn_arch} shows a generic belief network as a graphical model with each state influenced by its ancestors.
In the case of SBNs however, all states are binary and the activation function is the sigmoid function. They are one of the earliest neural networks used for generative modelling, predating all other presented models in this article.
%As described in the overview above, SBNs contain a number of binary states $s$, most commonly divided into many layers.
Each state $s_i$ is a random variable in the bayesian framework which means it can be observed, hidden or unknown representing different states of knowledge. As such it can also be interpreted with a probability distribution.
\begin{equation}
  \label{eq:sbn_node}
p(s_i) = \sigma\bigg(\sum_{j<i}W_{i,j}s_j+b_i\bigg)
\end{equation}
In equation \ref{eq:sbn_node}, $\sigma$ denotes the sigmoid function $\frac{1}{1 + e^{-x}}$ and $W_{i,j}$ the connection weights from state $s_j$ to $s_i$.

To draw samples from this network, it is possible to simply evaluate equation \ref{eq:sbn_node} for each visible node.
However prior to that, the SBN first needs to learn how to generate good samples which includes the \emph{learning} and the \emph{inference} problem.\\
Learning means in this context to adjust the weights and biases of the network to be more likely to generate previously observed input.
Meanwhile statistical inference is tasked to infer the states of the hidden nodes based on the input.\\

Due to each state being statistically depend on all ancestors, inference means to evaluate all possible configurations of every previous node.
Especially in deep networks, this kind of computation is infeasible as it requires evaluating exponentially many configurations.\\

% DONE

%However in order to draw good samples from this network, the weights need to be adjusted to make it 
%However most other operations involving this network are intractable for most cases, like learning and performing inference.

%In probabilistic networks, we are mostly interested in the aspects: sampling, learning and inferring.
%Even though sampling can be done efficiently, the other operations are intractable in almost all cases.
%\paragraph{Sampling}
%can be done using ancestral sampling through the hidden layers evaluating \ref{eq:sbn_node}

%These random variables are interpreted in a bayesian way where each one may be observed, hidden (or latent) or unknown representing different states of knowledge or belief \cite{definetti:1974}.

%\newpage
%\subsection{Learning}
%\begin{figure}[htb]
%\centering
%\includestandalone[mode=buildnew]{media/sbn_node}
  %\caption{SBN node}
  %\label{fig:sbn_node}
%\end{figure}


%In its most common form, SBNs are comprised by a number of layers where each node in a layer is a sigmoid function of its ancestors:

%While sampling from these networks is very efficient, computing the inference $p(z|x)$ over the hidden units is intractable.
%% show why?

%There have been several attempts on solving the inference problem, using inference networks which have shown promising results.

\newpage


%----------------------------------
% Variational Autoencoders
%----------------------------------
\section{Variational Autoencoders}
\label{sec:vae}
Variational Autoencoders (VAE) have been introduced in 2014 by Kingma and Welling which uses variational inference to perform efficient inference and learning in deep latent models.
% maybe extra section/sub?

\paragraph{Latent Space}
We assume that the high-dimensional input data $x$ can be explained using low-dimensional latent variables $z$.\\
While this might be surprising at first, most high-dimensional data we encounter can be explained using far fewer dimensions.
Let us take a look at different examples:\\
\begin{itemize}
  \item \textbf{Images} often have multiple thousands of dimensions, but the underlying structure is often far simpler. For example are 28x28 images of handwritten digits 784-dimensional but assuming one-hot encoding the most important information - the digit - it can be explained using 10 dimensions. There are a few more dimensions, stroke width, cursive, position, size. But still far lower than 784. We will take a look at this example later on in \ref{sec:vae}.
  \item \textbf{Speech} or more precisely the recording of it is high-dimensional, while the information it carries (the spoken words) can be represented in a low-dimensional space.
  \item \textbf{...}
\end{itemize}

Given this assumption, building a good representation in the latent space is crucial for minimal reconstruction error as well as good generative modelling (?).
%The VAE framework provides a directed model which aims to infer low-dimensional latent variables $z$ from high-dimensional input data $x$ and reconstruct the input.
\paragraph{Formal Setup}
Formally, $p(x,z)$ is the joint probability distribution over both input and latent variables while $p(z|x)$ is the conditional probability of the latent variables given input data.
Inferring the posterior distribution $p(z|x)$ is particularly interesting, because it means to enable learning parameters for good latent space representation.
$p(z|x)$ can be expanded using bayes rule to $\frac{p(x|z) p(z)}{p(x)}$.
Computing the nominator is straightforward, $p(z)$ is the probability distribution we chose for the latent space, oftenmost simply a gaussian.
$p(x|z)$ is easy to compute as well, ...? (forward pass?).\\
Meanwhile is the denominator difficult to evaluate, because it requires to consider all possible input combinations.
What we do instead in the variational autoencoders is to derive a lower-bound on $p(z|x)$ using a auxiliary distribution called $q_\theta(z|x)$. The subscript $\theta$ indicates that the distribution is parametrized by an variational term, so that for one $\theta$, $q_\theta(z|x) \approxeq p(z|x)$ holds ($q(z|x)$ doesn't need to depend on $x$).
%Performing statistical inference means computing $p(z|x)$ which is intractable in all but very simple cases (integrating over it, show calc).

%In this setup we have observed variables $x$ from which we would like to infer latent variables $z$


%Variational Autoencoders (VAE) have been a popular choice for unsupervised learning of complicated distributions (citation needed) and generative modelling.
%In order to generate data from unknown and mostly intractable distributions, we need approximations.
%There are basically two approaches for sampling from these distributions,
%first there are approximate samples (MCMC, gibbs sampling, etc) which try to directly approximate $p(x)$.
%Variational Autoencoders instead try to match an easier to compute distribution $q(x)$ to $p(x)$.
%By using this approach, VAEs are computationally less intensive (citation?) but have the drawback of being more restricted in their modelling approach.
%Practically, this means that with more computation MCMC methods approach $p(x)$, while there is no such guarantee for variational methods (--> not for all).

%VAE can be learned with just backpropagation (paper,TODO), but they differ from denoising and sparse autoencoders due to the different loss function.


%VAEs are built on top of neural networks and are designed in a way to allow training with gradient-based methods.
%Learning and inference are reasonable efficient and relatively easy to implement and show decent results, but have been overshadowed by more recent adversarial approaches (citation needed!!,see \ref{sec:gan}).

\subsection{Architecture}
\label{sub:vae_architecture}

\begin{figure}[htb]
\centering
\resizebox{5cm}{!}{\documentclass{standalone}
\usepackage{tikz}
\definecolor{cdfdfdf}{RGB}{223,223,223}
\definecolor{cffffff}{RGB}{255,255,255}

\begin{document}
\usetikzlibrary{arrows.meta}

\begin{tikzpicture}[y=0.80pt, x=0.80pt, yscale=-1.000000, xscale=1.000000, inner sep=0pt, outer sep=0pt]
\begin{scope}[cm={{1.25,0.0,0.0,-1.25,(0.0,1052.3622)}}]
  \begin{scope}[shift={(306.491,697.587)}]
    \path[draw=black,fill=cdfdfdf,line join=miter,line cap=butt,miter
      limit=10.00,nonzero rule,line width=0.319pt] (9.9628,0.0000) .. controls
      (9.9628,5.5023) and (5.5023,9.9628) .. (0.0000,9.9628) .. controls
      (-5.5023,9.9628) and (-9.9628,5.5023) .. (-9.9628,0.0000) .. controls
      (-9.9628,-5.5023) and (-5.5024,-9.9628) .. (0.0000,-9.9628) .. controls
      (5.5023,-9.9628) and (9.9628,-5.5024) .. (9.9628,0.0000) -- cycle;
  \end{scope}
    \path[cm={{1.0,0.0,0.0,-1.0,(303.468,695.373)}},fill=black,nonzero rule]
      (0.0000,0.0000) node[above right] (text4171) {$x$};
  \begin{scope}[shift={(306.491,697.587)}]
    \path[draw=black,fill=cffffff,line join=miter,line cap=butt,miter
      limit=10.00,nonzero rule,line width=0.319pt] (9.9628,48.6708) .. controls
      (9.9628,54.1732) and (5.5024,58.6336) .. (0.0000,58.6336) .. controls
      (-5.5023,58.6336) and (-9.9628,54.1732) .. (-9.9628,48.6708) .. controls
      (-9.9628,43.1685) and (-5.5024,38.7081) .. (0.0000,38.7081) .. controls
      (5.5023,38.7081) and (9.9628,43.1685) .. (9.9628,48.6708) -- cycle;
  \end{scope}
    \path[cm={{1.0,0.0,0.0,-1.0,(303.945,744.043)}},fill=black,nonzero rule]
      (0.0000,0.0000) node[above right] (text4181) {$z$};
    \path[cm={{1.0,0.0,0.0,-1.0,(260.848,741.767)}},fill=black,nonzero rule]
      (0.0000,0.0000) node[above right] (text4187) {$\phi$};
    \path[cm={{1.0,0.0,0.0,-1.0,(345.199,742.798)}},fill=black,nonzero rule]
      (0.0000,0.0000) node[above right] (text4193) {$\theta$};
  \begin{scope}[shift={(306.491,697.587)}]
    \path[draw=black,dash pattern=on 2.39pt off 2.39pt,line join=miter,line
      cap=butt,miter limit=10.00,line width=0.319pt] (-38.5088,48.6708) --
      (-10.9710,48.6708);
  \end{scope}
  \begin{scope}[shift={(295.52,746.258)}]
    \path[draw=black,fill=black,line join=miter,line cap=butt,miter
      limit=10.00,nonzero rule,line width=0.319pt] (-5.2035,2.3356) --
      (0.2989,0.0000) -- (-5.2035,-2.3356) -- (-5.2035,2.3356) -- cycle;
  \end{scope}
  \begin{scope}[shift={(306.491,697.587)}]
    \path[draw=black,line join=miter,line cap=butt,miter limit=10.00,line
      width=0.319pt] (38.5088,48.6708) -- (10.9710,48.6708);
  \end{scope}
  \begin{scope}[cm={{-1.0,0.0,0.0,-1.0,(317.462,746.258)}}]
    \path[draw=black,fill=black,line join=miter,line cap=butt,miter
      limit=10.00,nonzero rule,line width=0.319pt] (-5.2035,2.3356) --
      (0.2989,0.0000) -- (-5.2035,-2.3356) -- (-5.2035,2.3356) -- cycle;
  \end{scope}
  \begin{scope}[shift={(306.491,697.587)}]
    \path[draw=black,line join=miter,line cap=butt,miter limit=10.00,line
      width=0.319pt] (38.5088,45.5084) -- (7.1087,8.4005);
  \end{scope}
  \begin{scope}[cm={{-0.64792,-0.7657,0.7657,-0.64792,(313.6,705.988)}}]
    \path[draw=black,fill=black,line join=miter,line cap=butt,miter
      limit=10.00,nonzero rule,line width=0.319pt] (-5.2035,2.3356) --
      (0.2989,0.0000) -- (-5.2035,-2.3356) -- (-5.2035,2.3356) -- cycle;
  \end{scope}
  \begin{scope}[shift={(306.491,697.587)}]
    \path[draw=black,dash pattern=on 2.39pt off 2.39pt,line join=miter,line
      cap=butt,miter limit=10.00,line width=0.319pt] (-7.1856,7.1856) .. controls
      (-16.6444,16.6438) and (-16.6444,32.0271) .. (-7.7582,40.9132);
  \end{scope}
  \begin{scope}[cm={{0.7071,0.7071,-0.7071,0.7071,(298.733,738.5)}}]
    \path[draw=black,fill=black,line join=miter,line cap=butt,miter
      limit=10.00,nonzero rule,line width=0.319pt] (-5.2035,2.3356) --
      (0.2989,0.0000) -- (-5.2035,-2.3356) -- (-5.2035,2.3356) -- cycle;
  \end{scope}
  \begin{scope}[shift={(306.491,697.587)}]
    \path[draw=black,line join=miter,line cap=butt,miter limit=10.00,line
      width=0.319pt] (0.0000,38.5088) -- (0.0000,10.9710);
  \end{scope}
  \begin{scope}[cm={{0.0,-1.0,1.0,0.0,(306.491,708.558)}}]
    \path[draw=black,fill=black,line join=miter,line cap=butt,miter
      limit=10.00,nonzero rule,line width=0.319pt] (-5.2035,2.3356) --
      (0.2989,0.0000) -- (-5.2035,-2.3356) -- (-5.2035,2.3356) -- cycle;
  \end{scope}
    \path[cm={{1.0,0.0,0.0,-1.0,(308.318,675.918)}},fill=black,nonzero rule]
      (0.0000,0.0000) node[above right] (text4239) {N};
  \begin{scope}[shift={(306.491,697.587)}]
    \path[draw=black,line join=miter,line cap=butt,miter limit=10.00,line
      width=0.319pt] (16.5376,62.3527) -- (-16.5376,62.3527) .. controls
      (-18.7386,62.3527) and (-20.5228,60.5685) .. (-20.5228,58.3676) --
      (-20.5228,-21.2037) .. controls (-20.5228,-23.4046) and (-18.7386,-25.1888) ..
      (-16.5376,-25.1888) -- (16.5376,-25.1888) .. controls (18.7386,-25.1888) and
      (20.5228,-23.4046) .. (20.5228,-21.2037) -- (20.5228,58.3676) .. controls
      (20.5228,60.5685) and (18.7386,62.3527) .. (16.5376,62.3527) -- cycle;
  \end{scope}
\end{scope}

\end{tikzpicture}
\end{document}
}
  \caption{VAE architecture (source: VAE paper)}\label{fig:vae_architecture}
\end{figure}
VAE just like other autoencoders encode the input data into a latent space similar to compression of data and is able to decode a vector of latent variables into output while trying to match the output to the input.
But in contrast to other autoencoders (sparse, denoising), we enforce a specific distribution on the latent space.
This allows to sample from this distribution and generate output which will look similar to the data on which the VAE has been trained.
\ref{fig:vae_architecture}.

\paragraph{Relation to Auto-Encoders}
The Variational autoencoder has the same overall architecture than other autoencoders, for example sparse or denoising autoencoders.
Similar to other autoencoder frameworks VAE provides an encoder as well as a decoder model, but in contrast to other autoencoders both networks are probabilistic. This means that the model described by the VAE framework can be seen as a jointly trained probabilistic encoder and probabilistic decoder.
%One of the main differences is that the VAE framework enforces a specific prior distribution on the latent space, most of the time simply an isotropic gaussian.



\paragraph{The Kullback-Leibler Divergence} (KL divergence) is a measurement for the difference between two probability distributions. $\mathcal{D}_{\mathrm{KL}}(P \| Q)$ can informally described as the amount of information which is lost when using $Q$ to represent $P$.
The KL divergence does not obey the triangle inequality and is also not symmetrical, therefore it does not qualify as a metric.
$\mathcal{D}_{\mathrm{KL}}(P||Q)$ can nevertheless be understood as a measure of the difference between $P$ and $Q$ and is as such used in the VAE to approximate the true posterior distribution.

\subsection{Objective Function}
The objective function of the VAE consists of the reconstruction error and the regularizer.
%$\mathcal{L}$ learns both the encoder as well as the decoder with their respective parameters $\phi$ and $\theta$.
% probabilistic encoder $q_\phi(z|x)$ (produces z values from which x could've been generated)
% probabilistic decoder $p_\theta(x|z)$
We will discuss both terms in detail below following the derivation of the lower variational bound $\mathcal{L}$ and the rewritten objective function.

\paragraph{Derivation of lower bound}
\begin{align*}
  \log p(x)\\
  &= \int_z q_\phi(z|x) \log p(x)\\
  &= \int_z q_\phi(z|x) \log \frac{p_\theta(x,z)}{p_\theta(z|x)} \tag{Bayes' rule}\\
  &= \int_z q_\phi(z|x) \log\bigg(\frac{p_\theta(z,x)}{q_\phi(z|x)} \frac{q_\phi(z|x)}{p_\theta(z|x)}\bigg) \tag{Chain rule}\\
  &= \int_z q_\phi(z|x) \bigg(\log\frac{p_\theta(z,x)}{q_\phi(z|x)} + \log\frac{q_\phi(z|x)}{p_\theta(z|x)}\bigg)\\
  &= \int_z q_\phi(z|x) \log \frac{p_\theta(z,x)}{q_\phi(z|x)} + \int_z q_\phi(z|x) \log\frac{q_\phi(z|x)}{p_\theta(z|x)}\\
  &= \underbrace{\int_z q_\phi(z|x) \log \frac{p_\theta(z,x)}{q_\phi(z|x)}}_{= \mathcal{L}}+ \underbrace{\mathcal{D}_{\mathrm{KL}}\big(q_\phi(z|x) \| p_\theta(z|x)\big)}_{\geq 0}\\
  &\geq \mathcal{L}\\
\end{align*}

$\mathcal{L}$ is the lower bound to $\log p(x)$.\\
In order to get the loss function for individual data points or minibatches we rewrite the lower bound.

\begin{align*}
  \mathcal{L}
  &= \int_z q_\phi(z|x) \log\frac{p_\theta(z,x)}{q_\phi(z|x)}\\
  &= \int_z q_\phi(z|x) \log\frac{p(z) p_\theta(x|z)}{q_\phi(z|x)} \tag{we assume that x is conditionally dependent on z}\\
  &= \int_z q_\phi(z|x) \bigg(\log\frac{p(z)}{q_\phi(z|x)} + \log p_\theta(x|z)\bigg)\\
  &= -\int_z q_\phi(z|x) \log\frac{q_\phi(z|x)}{p(z)} + \int_z q_\phi(z|x) \log p_\theta(x|z)\\
  &= -\mathcal{D}_{\mathrm{KL}}\big(q_\phi(z|x) \| p(z)\big) + \int_z q_\phi(z|x) \log p_\theta(x|z)\\
  &= \underbrace{-\mathcal{D}_{\mathrm{KL}}\big(q_\phi(z|x) \| p(z)\big)}_{\mathcal{L}_{reg}} + \underbrace{\mathbb{E}_{z \sim q_\phi(z|x)}\big[ \log p_\theta(x|z)\big]}_{\mathcal{L}_{rec}}\\
  &= \mathcal{L}_{reg} + \mathcal{L}_{rec}
\end{align*}


\paragraph{Regularization term $\mathcal{L}_{reg}$} encourages the model to learn simple representations in latent space, due to the negative Kullback-Leibler (KL) divergence between the learned variational distribution $q_\theta(z|x)$ and the latent space prior distribution $p(z)$.

% move to architecture:
%In the original VAE paper, the unit gaussian distribution with diagonal variance was proposed for $p(z)$ and although there have been multiple proposals (citations, multi-modal etc) for different distributions we will stick with $\mathcal{N}(0,I)$ for the sake of simplicity.
% ---

%For two probability distributions $P$ and $Q$, the KL divergence $\mathcal{D}_{\mathrm{KL}}(P||Q)$ can informally be described as the amount of information which is lost when using $Q$ to represent $P$.

\paragraph{Reconstruction error $\mathcal{L}_{rec}$} measures how well the decoder reconstructs the input data using the latent space $z$.
This error is measures using the negative log-likelihood of the conditional probability distribution $p_\theta(x_i|z)$ where $z$ is sampled from $q_\phi(z | x_i)$.
$\mathcal{L}_{\mathrm{rec}}$ is necessary to force the encoder to produce latent variables which can be used to reconstruct the input data well.




\subsubsection{Reparameterization Trick}
To be able to backpropagate the loss function through the VAE, it has to be differentiable and deterministic.
Because we add noise to the encoding, the gradients can't be computed directly. In order to circumvent this restriction, the so-called "reparameterization trick" is applied.
Instead of drawing $z ~ \mathcal{N}(\mu(x), \Sigma(x))$, we sample an auxiliary variable $\epsilon$ from $\mathcal{N}(0, I)$ which we then transform with equation \ref{eq:rep_trick} into $z$.
\begin{equation}
  \label{eq:rep_trick}
  z = \mu(x) + \Sigma^{1/2}(x)*\epsilon
\end{equation}
This allows us to compute the gradient of the loss function and backpropagate through the entire model and only have the stochastic variable $\epsilon$ as input.\\\\

The core idea of VAEs is to match a distribution $q(x)$ to the desired $p(x)$ by enforcing a lower variational bound on $q(x)$ in a way that it seeks to match $p(x)$.
In contrast to other autoencoders, VAEs behave differently due to this lower bound but they resemble the architecture of traditional autoencoders (sparse, denoising).




We use VAEs when we have a complicated distribution $p_\theta(x,z)$ with unknown latent variables $z$.
The prior distribution $p_\theta(z)$ over the latent structure is a centered isotropic(?) multivariate Gaussian denoted by $\mathcal{N}(z;0, I)$.
The constructed architecture uses an probabilistic encoder $q_\theta(z|x)$ and a probabilistic decoder $p_\theta(x|z)$ in a form of neural networks, in the original paper MLP are used\cite{vae:2013} but there are various extensions (see \ref{sub:vae_extensions}, TODO).
Because the posterior is intractable ($p_\theta(x)$), we instead approximately maximize the lower variational bound $L(\theta,\phi;x)$.\\
\begin{equation}
  \mathcal{L}(\theta,\phi;x) = -D_{KL}(q_\theta(z|x)||p_\theta(z)) + \mathbb{E}_{q_\theta(z|x)}[log_{p_\theta}(x|z)]
\end{equation}




\subsection{Learning}
\label{sub:vae_learning}
In practice, the usual choice for $q_\theta(z|x)$ is the multivariate gaussian distribution.
$$
q_\theta(z|x) = \mathcal{N}(z|\mu(X;\vartheta), \Sigma(X;\vartheta))
$$
Because we assume that $P(z)$ is also a multivariate Gaussian distribution, the KL-divergence can now be computed in closed form as follows\cite{derivations:2007}.
\begin{align*}
  \mathcal{D}_{\mathrm{KL}}\big[Q(z) || P(z|x)\big] &= \mathbb{E}\big[\log \frac{p}{q}\big]\\
  \mathcal{D}_{\mathrm{KL}}\big[\mathcal{N}(\mu_0,\Sigma_0) || \mathcal{N}(\mu_1,\Sigma_1)\big]
  &= \frac{1}{2} \big(\mathrm{tr}\big(\Sigma_1^{-1}\Sigma_0\big) + \big(\big)\big)
\end{align*}


\subsection{Inference}
\label{sub:vae_inference}
\begin{figure}[htb]
\centering
%\resizebox{5cm}{!}{\documentclass{standalone}
\usepackage{tikz}
\definecolor{cdfdfdf}{RGB}{223,223,223}
\definecolor{cffffff}{RGB}{255,255,255}

\begin{document}
\usetikzlibrary{arrows.meta}

\begin{tikzpicture}[y=0.80pt, x=0.80pt, yscale=-1.000000, xscale=1.000000, inner sep=0pt, outer sep=0pt]
\begin{scope}[cm={{1.25,0.0,0.0,-1.25,(0.0,1052.3622)}}]
  \begin{scope}[shift={(306.491,697.587)}]
    \path[draw=black,fill=cdfdfdf,line join=miter,line cap=butt,miter
      limit=10.00,nonzero rule,line width=0.319pt] (9.9628,0.0000) .. controls
      (9.9628,5.5023) and (5.5023,9.9628) .. (0.0000,9.9628) .. controls
      (-5.5023,9.9628) and (-9.9628,5.5023) .. (-9.9628,0.0000) .. controls
      (-9.9628,-5.5023) and (-5.5024,-9.9628) .. (0.0000,-9.9628) .. controls
      (5.5023,-9.9628) and (9.9628,-5.5024) .. (9.9628,0.0000) -- cycle;
  \end{scope}
    \path[cm={{1.0,0.0,0.0,-1.0,(303.468,695.373)}},fill=black,nonzero rule]
      (0.0000,0.0000) node[above right] (text4171) {$x$};
  \begin{scope}[shift={(306.491,697.587)}]
    \path[draw=black,fill=cffffff,line join=miter,line cap=butt,miter
      limit=10.00,nonzero rule,line width=0.319pt] (9.9628,48.6708) .. controls
      (9.9628,54.1732) and (5.5024,58.6336) .. (0.0000,58.6336) .. controls
      (-5.5023,58.6336) and (-9.9628,54.1732) .. (-9.9628,48.6708) .. controls
      (-9.9628,43.1685) and (-5.5024,38.7081) .. (0.0000,38.7081) .. controls
      (5.5023,38.7081) and (9.9628,43.1685) .. (9.9628,48.6708) -- cycle;
  \end{scope}
    \path[cm={{1.0,0.0,0.0,-1.0,(303.945,744.043)}},fill=black,nonzero rule]
      (0.0000,0.0000) node[above right] (text4181) {$z$};
    \path[cm={{1.0,0.0,0.0,-1.0,(260.848,741.767)}},fill=black,nonzero rule]
      (0.0000,0.0000) node[above right] (text4187) {$\phi$};
    \path[cm={{1.0,0.0,0.0,-1.0,(345.199,742.798)}},fill=black,nonzero rule]
      (0.0000,0.0000) node[above right] (text4193) {$\theta$};
  \begin{scope}[shift={(306.491,697.587)}]
    \path[draw=black,dash pattern=on 2.39pt off 2.39pt,line join=miter,line
      cap=butt,miter limit=10.00,line width=0.319pt] (-38.5088,48.6708) --
      (-10.9710,48.6708);
  \end{scope}
  \begin{scope}[shift={(295.52,746.258)}]
    \path[draw=black,fill=black,line join=miter,line cap=butt,miter
      limit=10.00,nonzero rule,line width=0.319pt] (-5.2035,2.3356) --
      (0.2989,0.0000) -- (-5.2035,-2.3356) -- (-5.2035,2.3356) -- cycle;
  \end{scope}
  \begin{scope}[shift={(306.491,697.587)}]
    \path[draw=black,line join=miter,line cap=butt,miter limit=10.00,line
      width=0.319pt] (38.5088,48.6708) -- (10.9710,48.6708);
  \end{scope}
  \begin{scope}[cm={{-1.0,0.0,0.0,-1.0,(317.462,746.258)}}]
    \path[draw=black,fill=black,line join=miter,line cap=butt,miter
      limit=10.00,nonzero rule,line width=0.319pt] (-5.2035,2.3356) --
      (0.2989,0.0000) -- (-5.2035,-2.3356) -- (-5.2035,2.3356) -- cycle;
  \end{scope}
  \begin{scope}[shift={(306.491,697.587)}]
    \path[draw=black,line join=miter,line cap=butt,miter limit=10.00,line
      width=0.319pt] (38.5088,45.5084) -- (7.1087,8.4005);
  \end{scope}
  \begin{scope}[cm={{-0.64792,-0.7657,0.7657,-0.64792,(313.6,705.988)}}]
    \path[draw=black,fill=black,line join=miter,line cap=butt,miter
      limit=10.00,nonzero rule,line width=0.319pt] (-5.2035,2.3356) --
      (0.2989,0.0000) -- (-5.2035,-2.3356) -- (-5.2035,2.3356) -- cycle;
  \end{scope}
  \begin{scope}[shift={(306.491,697.587)}]
    \path[draw=black,dash pattern=on 2.39pt off 2.39pt,line join=miter,line
      cap=butt,miter limit=10.00,line width=0.319pt] (-7.1856,7.1856) .. controls
      (-16.6444,16.6438) and (-16.6444,32.0271) .. (-7.7582,40.9132);
  \end{scope}
  \begin{scope}[cm={{0.7071,0.7071,-0.7071,0.7071,(298.733,738.5)}}]
    \path[draw=black,fill=black,line join=miter,line cap=butt,miter
      limit=10.00,nonzero rule,line width=0.319pt] (-5.2035,2.3356) --
      (0.2989,0.0000) -- (-5.2035,-2.3356) -- (-5.2035,2.3356) -- cycle;
  \end{scope}
  \begin{scope}[shift={(306.491,697.587)}]
    \path[draw=black,line join=miter,line cap=butt,miter limit=10.00,line
      width=0.319pt] (0.0000,38.5088) -- (0.0000,10.9710);
  \end{scope}
  \begin{scope}[cm={{0.0,-1.0,1.0,0.0,(306.491,708.558)}}]
    \path[draw=black,fill=black,line join=miter,line cap=butt,miter
      limit=10.00,nonzero rule,line width=0.319pt] (-5.2035,2.3356) --
      (0.2989,0.0000) -- (-5.2035,-2.3356) -- (-5.2035,2.3356) -- cycle;
  \end{scope}
    \path[cm={{1.0,0.0,0.0,-1.0,(308.318,675.918)}},fill=black,nonzero rule]
      (0.0000,0.0000) node[above right] (text4239) {N};
  \begin{scope}[shift={(306.491,697.587)}]
    \path[draw=black,line join=miter,line cap=butt,miter limit=10.00,line
      width=0.319pt] (16.5376,62.3527) -- (-16.5376,62.3527) .. controls
      (-18.7386,62.3527) and (-20.5228,60.5685) .. (-20.5228,58.3676) --
      (-20.5228,-21.2037) .. controls (-20.5228,-23.4046) and (-18.7386,-25.1888) ..
      (-16.5376,-25.1888) -- (16.5376,-25.1888) .. controls (18.7386,-25.1888) and
      (20.5228,-23.4046) .. (20.5228,-21.2037) -- (20.5228,58.3676) .. controls
      (20.5228,60.5685) and (18.7386,62.3527) .. (16.5376,62.3527) -- cycle;
  \end{scope}
\end{scope}

\end{tikzpicture}
\end{document}
}
\includegraphics{media/manifold.pdf}
  \caption{MNIST 2-dimensional manifold}\label{fig:vae_manifold}
\end{figure}

\subsection{Performance}
\label{sub:vae_performance}

\subsection{Extensions}
\label{sub:vae_extensions}

\subsubsection{Deep Recurrent Attention Writer}
\label{ssub:vae_deep_recurrent_attention_writer}

\subsubsection{Importance Weighted Autoencoder}
\label{ssub:vae_importance_weighted_autoencoder}

\subsubsection{Conditional VAE}
\label{ssub:vae_conditional_vae}





%----------------------------------
% Generative Adversarial Netoworks
%----------------------------------
\section{Generative Adversarial Networks}
\label{sec:gan}

Generative Adversarial Networks (GAN) is a framework recently proposed by Goodfellow et al which has gathered a lot of attention in the deep learning community and in the last few years many extensions have been [proposed] (citations).
GANs can be understood as a two-player game where one player is called generator $G$ playing against the discriminator $D$.
The goal of $G$ is to produce similar data to the given training set while $D$ is tasked with discriminating(?) the generated samples from the training data. Both $G$ and $D$ are learning function approximators, for example Multi-Layer Peceptron (MLP) networks as proposed in the original GAN paper.
% --> declare distribution p_data, p_model, x, z
During learning, both networks are updated in parallel according to the gradient of their respective loss function which we will explore now.\\
In the following $p_{data}$ is the true data generating distribution from which the training samples have been generated.
We don't have access to that distribution.

\subsection{Architecture}
\label{sub:gan_arch}


\subsubsection{Generator Network}
The generator network in the GAN framework tries to fool the discriminator $D$ by producing samples which are indistinguishable from the training data.
This process can be formally written as minimizing the following cost function.
$$
\mathcal{C}_g = \log(1 - D(G(z)))
$$
$z$ is produced by sampling from the noise prior $p_g(z)$.\\

Note that this loss function is fully differentiable which means it can be trained with backpropagation.



\subsubsection{Discriminator Network}
The objective of the discriminator network $D$ is to distinguish the generated samples produced by $G$ from the training set samples.
This process can be formalized to maximizing the cost function $\mathcal{C}_d$.
$$
\mathcal{C}_d = \log\big(\log D(x) + \log (1 - D(G(z))\big)
$$



\subsection{Training}
\label{sub:gan_training}
The GAN framework can be fully trained with stochastic backpropagation due to both $D$ and $G$ being differentiable.

Combining the cost functions of generator and discriminator yield us the following objective with value function $V(G,D)$.
$$
\min_G \max_D V(D,G) = \mathbb{E}_{x \sim p_{data}}[\log D(x)] + \mathbb{E}_{z \sim p_z(z)}[\log(1 - D(G(z)))]
$$
In the original GAN paper an minibatch-based algorithm for training GAN is proposed, but we will take a look at an the online learning version below:\\
\begin{algorithm}
  \caption{Online learning of generative adversarial networks $-$ simple version ($k=1$)}
  \label{alg:gan_online}
  \begin{algorithmic}[1]
    \Let{$\eta_d, \eta_g$}{Learning rate of discriminator and generator networks, respectively}
    \Let{$\theta_d, \theta_g$}{Parameters of function approximators}
    \For{number of training iterations}
      \Let{$z_d$}{sample from noise prior $p_g(z)$}
      \Let{$z_g$}{sample from noise prior $p_g(z)$}
      \Let{$x$}{example from training set}
      \Let{$\theta_d$}{$\theta_d + \eta_d \nabla_{\theta_d} \bigg[\log D(x) + \log \big(1 - D(G(z_d))\big)\bigg]$}
      \Let{$\theta_g$}{$\theta_g - \eta_g \nabla_{\theta_g} \bigg[\log\big(1 - D(G(z_d))\big)\bigg]$}
    \EndFor
  \end{algorithmic}
\end{algorithm}

In algorithm \ref{alg:gan_online} the original algorithm has been modified for single pass (online) learning.
Sampling minibatches from the noise prior and $p_{data}(x)$ result in the minibatch version.
Additionally, it is often required for practical reasons to let the discriminator run multiple backward passes while updating the generator once (why?).
Note that the discriminator will ascend while the generator descend its gradient due to maximizing $\mathcal{C}_d$ but minimizing $\mathcal{C}_g$ (correct?).

\subsection{Stability and Performance}
\label{sub:gan_stability}

\subsubsection{Freeze Learning}
\label{ssub:gan_freeze_learning}

\subsubsection{Feature Matching}
\label{ssub:gan_feature_matching}

\subsubsection{Minibatch discrimination}
\label{ssub:gan_minibatch_discrimination}

\subsubsection{Historical Averaging}
\label{ssub:gan_historical_averaging}


\subsubsection{Label Smoothing}
\label{ssub:gan_label_smoothing}


\subsubsection{Normalizing Flows}
\label{ssub:gan_nf}









\subsection{Application}
\label{sub:gan_application}
GAN have been applied to many different areas which include generating images (GAN, DCGAN, LAPGAN), sequences (SeqGAN), videos (\cite{gan_video:2016}), 3D objects (\cite{gan_3d:2016}), text-to-image synthesis (\cite{gan_t2i:2016})
\subsection{Extensions}
\label{sub:gan_extensions}

\subsubsection{LAPGAN}
\label{ssub:lapgan}

\subsubsection{DCGAN}
\label{ssub:dcgan}





%----------------------------------
% More Generative Directed Models
%----------------------------------
\section{More Generative Directed Models}
\label{sec:more}

\subsection{Generative Moment Matching Networks}
\label{sub:more_mmn}

\subsection{Auto-Regressive Networks}
\label{sub:more_arn}

See \cite{gan_openai:2016} and \cite{gan_training:2016}.


\section{Conclusion}
\label{sec:conclusion}

Generative models have shown great promise to tackle interesting problems.
Academic work is moving incredibly fast and many new ideas and combinations
are being proposed almost daily (citation needed, meta paper?).
For example, VAEs and GANs have been combined (https://github.com/skaae/vaeblog)
which produced way larger images than GANs alone.

% not relevant, but wanted to write about it :p
As mentioned in the motivation, further advances to generative architectures are
likely to solve current issues with depending on large labelled data.
Another large and increasing area of research is semi-supervised learning
(combining learning from labelled data as well as unlabelled data)
and also one-shot learning, which is where humans excel and machines fail currently.
\\\\
Exciting times :)

\addcontentsline{toc}{section}{References}
\bibliographystyle{plain}
\bibliography{paper}

\end{document}
